\documentclass[prd]{revtex4}

\pdfoutput=1

\usepackage{graphicx}
%\usepackage{dcolumn}
%\usepackage{bm}
\usepackage{amssymb,amsmath,bm}  
%\usepackage{multirow}
%\usepackage[utf8]{inputenc}
%\usepackage{balance}
%\usepackage{enumitem}

%\usepackage{array}
%\newcolumntype{C}[1]{>{\centering\let\newline\\\arraybackslash\hspace{0pt}}m{#1}}
\usepackage{lipsum}

\usepackage[usenames,dvipsnames]{xcolor}
\usepackage{hyperref}
\hypersetup{
    colorlinks = true,
    citecolor = {MidnightBlue},
    linkcolor = {BrickRed},
    urlcolor = {BrickRed}
}
\usepackage{lipsum}  


\newcommand{\TIB}[1]{\textcolor{blue}{#1}}
\newcommand{\TB}[1]{\textcolor{red}{#1}}

\newcommand{\uKam}{\mu\text{K-arcmin}}
\newcommand{\jcap}{JCAP}
\newcommand{\mnras}{MNRAS}
\newcommand{\aap}{A\&A}
\newcommand{\araa}{ARA\&A}
\newcommand{\aaps}{A\&AS}
\newcommand{\apjs}{ApJS}
\newcommand{\apjl}{ApJL}
\newcommand{\physrep}{Physics Reports}
\newcommand{\aj}{Astron. Journal}
\newcommand{\pasp}{Publications of the ASP}
\newcommand{\procspie}{Proc. SPIE}
\newcommand{\bl}{{\bm \ell}}
\newcommand{\bx}{{\bm x}}
\newcommand{\ba}{\begin{eqnarray}}
\newcommand{\ea}{\end{eqnarray}}
\newcommand{\tj}[6]{ \begin{pmatrix}
   #1 & #2 & #3 \\
   #4 & #5 & #6 
\end{pmatrix}}


\begin{document}
\title{Combining the power spectra }

  \date{\today}
  \maketitle

Our goal is to combine spectra from different detector array and different season into a set of 'averaged' cross frequency spectra, the data model is the following
\ba
C_{b, \nu_{1} \nu_{2}}^{Xs_{1} \alpha , Y s_{2}  \beta } = P_{b,   \nu_{1}  \nu_{2}}^{XY} + \epsilon_{b, \nu_{1} \nu_{2}}^{Xs_{1} \alpha , Y s_{2}  \beta }
\ea
With $X,Y \in \{T,E\}$, $\nu_{1} ,\nu_{2}$ index frequencies,  $\alpha, \beta$ the detector arrays and $s_{1}, s_{2}$ the season of observation.
$P_{\ell,   \nu_{1}  \nu_{2}}^{XY} $ is what we would like to estimate. Assuming gaussiannity we can write a likelihood
\ba
-2 \ln {\cal L} \propto  \sum_{ \nu_{1} \nu_{2} \nu_{3} \nu_{4}} \sum_{ X, Y, W, Z} \sum_{ s_{1}s_{2}s_{3}s_{4}} \sum_{ \alpha \beta \gamma \mu} \sum_{bb'} \nonumber \\
  \left( C_{b, \nu_{1} \nu_{2}}^{Xs_{1} \alpha , Y s_{2}  \beta }  - \sum_{A, B, b_{a}, \nu_{a} \nu_{b}}  {\cal A}_{b,   \nu_{1}  \nu_{2}; b_{a} \nu_{a} \nu_{b}}^{Xs_{1} \alpha , Y s_{2}  \beta ; AB}  P_{b_{a},   \nu_{a}  \nu_{b}}^{AB} \right)^{T} \Xi^{-1}  \left( C_{b', \nu_{3} \nu_{4}}^{Ws_{3} \gamma , Z s_{4}  \mu }  - \sum_{E, F b_{a'}, \nu_{e} \nu_{f}}  {\cal A}_{b',   \nu_{3}  \nu_{4}; b_{a'} \nu_{e} \nu_{f}}^{Ws_{3} \gamma , Z s_{4}  \mu ; EF}  P_{b_{a'},   \nu_{e}  \nu_{f}}^{EF}\right)
\ea
This looks daunting only because we decided to write all index explicitly,  $ {\cal A}  $ associate the 'averaged' cross frequency spectra to the corresponding $ C_{b}$ element, it's a rectangular matrix with expression
\ba
  {\cal A}_{b,   \nu_{1}  \nu_{2}; b_{a} \nu_{a} \nu_{b}}^{Xs_{1} \alpha , Y s_{2}  \beta ; AB}   = \delta_{b, b_{a}}  \delta_{ \nu_{1}, \nu_{a}} \delta_{ \nu_{2}, \nu_{b}}   \delta^{X, A} \delta^{Y, B} 
\ea

and $\Xi$ is the usual covariance matrix. A maximum likelihood estimation of the  'averaged' cross frequency spectra is given by
$\frac{ \partial \ln {\cal L}}{\partial \bm{P}} |_{\hat{\bm{P}}}= 0$
We have
\ba
 - 2 {\cal A}^{T} \Xi^{-1}  \left( \bm{C}- {\cal A}   \bm{P} \right) = 0 \\
  \bm{P}  =   ({\cal A}^{T} \Xi^{-1} {\cal A} )^{-1} {\cal A}^{T} \Xi^{-1}  \bm{C}
\ea


\end{document}




